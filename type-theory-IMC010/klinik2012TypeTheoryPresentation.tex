\documentclass{beamer}

\beamertemplatenavigationsymbolsempty

\mode<presentation>
{
  \usetheme{default}
}

\usepackage[english]{babel}
\usepackage[latin1]{inputenc}
\usepackage{bussproofs}

% needs debian package texlive-math-extra
\usepackage{stmaryrd} % for \llbracket, \rrbracket

\usepackage{times}
\usepackage[T1]{fontenc}
% Or whatever. Note that the encoding and the font should match. If T1
% does not look nice, try deleting the line with the fontenc.

\usepackage{tikz}
\usetikzlibrary{positioning}
\usetikzlibrary{calc}

\title
{Syntax Directed Type Checking for Pure~Type~Systems}

\author
{Markus~Klinik}

\institute[Radboud University Nijmegen] % (optional, but mostly needed)
{
  Radboud University Nijmegen
}

\date
{Type Theory and Proof Assistants 2012}


\newcommand{\arr}{\rightarrow}
\newcommand{\Arr}{\Rightarrow}

\begin{document}

\begin{frame}
  \titlepage
\end{frame}

\begin{frame}{Outline}
  \tableofcontents
  % You might wish to add the option [pausesections]
\end{frame}


% Structuring a talk is a difficult task and the following structure
% may not be suitable. Here are some rules that apply for this
% solution:

% - Exactly two or three sections (other than the summary).
% - At *most* three subsections per section.
% - Talk about 30s to 2min per frame. So there should be between about
%   15 and 30 frames, all told.

% - A conference audience is likely to know very little of what you
%   are going to talk about. So *simplify*!
% - In a 20min talk, getting the main ideas across is hard
%   enough. Leave out details, even if it means being less precise than
%   you think necessary.
% - If you omit details that are vital to the proof/implementation,
%   just say so once. Everybody will be happy with that.

\section{Explaining the Title}

\subsection{What is Type Checking?}


\begin{frame}{What is Type Checking?}


\end{frame}


\subsection{What Does ``Syntax Directed'' Mean?}


\begin{frame}{What Does ``Syntax Directed'' Mean?}

 \begin{prooftree}
  \AxiomC{}
  %\RightLabel{[Unit]}
  \UnaryInfC{unity : unit}
 \end{prooftree}

 \begin{prooftree}
  \AxiomC{$\Gamma \vdash M : s$}
  \AxiomC{$\Gamma \vdash N : t$}
  %\RightLabel{$[Pairing]$}
  \BinaryInfC{$\Gamma \vdash (M, N) : s \times t$}
 \end{prooftree}

 \begin{prooftree}
  \AxiomC{$\Gamma \vdash M : s \times t$}
  %\RightLabel{$[First]$}
  \UnaryInfC{$\Gamma \vdash fst(M) : s$}
 \end{prooftree}

 \begin{prooftree}
  \AxiomC{$\Gamma \vdash M : s \times t$}
  %\RightLabel{$[Second]$}
  \UnaryInfC{$\Gamma \vdash snd(M) : t$}
 \end{prooftree}

\end{frame}


\subsection{What are PTSs?}

\begin{frame}{What are PTSs?}

  \begin{itemize}
    \item
      Generalized type systems
    \item
      Zoo $\Arr$ $\lambda$-cube $\Arr$ PTSs
  \end{itemize}

\end{frame}


\begin{frame}{The $\lambda$-Cube}

\end{frame}


\section{Syntax Directed Type Checking for PTSs}

\subsection{The Problem}

\begin{frame}{The Problem}

  \begin{itemize}
    \item
      Foo
    \item
      Bar
    \item
      Baz
  \end{itemize}

\end{frame}


\subsection{Step 1: The Weakening Rule}

\begin{frame}{The Weakening Rule}

  \begin{itemize}
    \item
      Foo
    \item
      Bar
    \item
      Baz
  \end{itemize}

\end{frame}


\subsection{Step 2: The Second Step}

\begin{frame}{The Second Step}

  \begin{itemize}
    \item
      Foo
    \item
      Bar
    \item
      Baz
  \end{itemize}

\end{frame}


\subsection{Step 3: The Third Step}

\begin{frame}{The Third Step}

  \begin{itemize}
    \item
      Foo
    \item
      Bar
    \item
      Baz
  \end{itemize}

\end{frame}


\subsection{Step 4: The Fourth Step}

\begin{frame}{The Fourth Step}

  \begin{itemize}
    \item
      Foo
    \item
      Bar
    \item
      Baz
  \end{itemize}

\end{frame}


\section*{Summary}

\begin{frame}{Summary}

  \begin{itemize}
  \item
    Conclusion A
  \item
    Conclusion B
  \item
    Conclusion C
  \end{itemize}

\end{frame}


\end{document}


