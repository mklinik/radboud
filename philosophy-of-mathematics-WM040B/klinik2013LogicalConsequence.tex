\documentclass[a4paper]{article}

% Font stuff
\usepackage{fouriernc}
%\usepackage{mathptmx}
\usepackage[T1]{fontenc}

% Extra math symbols
\usepackage{amsmath}
\usepackage{amssymb}
\usepackage{amsthm}

% IEEE equation array
\usepackage[retainorgcmds]{IEEEtrantools}

% bibliography
%\usepackage[square]{natbib}

% adds a toc to the pdf file, and makes refs clickable
\usepackage[bookmarks,colorlinks=true]{hyperref}

% for \llbracket, \rrbracket (scott brackets)
% Needs debian package texlive-math-extra
\usepackage{stmaryrd}

% === BEGIN custom commands

\newcommand{\arr}{\rightarrow}
\newcommand{\todo}[1]{\bigskip \noindent \emph{todo: #1}}
%\newcommand{\todo}[1]{}
\newcommand{\semantics}[1]{\llbracket #1 \rrbracket}

% church-style (explicit) abstraction. We adjust the spacing around the colon and dot
\newcommand{\church}[4]{#1 #2\!:\!#3\,.\,#4}

\newcommand{\curry}[3]{#1 #2\,.\,#3}

% something is of type something  x : A, we adjust the spacing
\newcommand{\oftype}[2]{#1\!:\!#2}

% === END custom commands

\begin{document}

\title{On Logical Consequence}
\author{Markus Klinik (s4220315)}
\maketitle

\begin{abstract}

This document is a bullet-point summary of two papers on logical consequence:
Tarski's ``On The Concept of Following Logically'' and Shapiro's ``Logical
Consequence, Proof Theory and Model Theory''.

\end{abstract}

\section{Tarksi's Paper}

\section{Shapiro's Paper}

\bibliographystyle{plain}
\bibliography{logic}

\end{document}

% vim: textwidth=80
