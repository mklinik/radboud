\documentclass[a4paper]{article}

% Font stuff
\usepackage{fouriernc}
%\usepackage{mathptmx}
\usepackage[T1]{fontenc}

% Extra math symbols
\usepackage{amsmath}
\usepackage{amssymb}
\usepackage{amsthm}

% IEEE equation array
\usepackage[retainorgcmds]{IEEEtrantools}

% bibliography
%\usepackage[square]{natbib}

% adds a toc to the pdf file, and makes refs clickable
\usepackage[bookmarks,colorlinks=true]{hyperref}

% for \llbracket, \rrbracket (scott brackets)
% Needs debian package texlive-math-extra
\usepackage{stmaryrd}

% proof trees
\usepackage{bussproofs}
\def\defaultHypSeparation{\hskip .1in}

% tikz drawing library
\usepackage{tikz}
\usetikzlibrary{positioning}
\usetikzlibrary{matrix}

% === BEGIN custom commands

\newcommand{\arr}{\rightarrow}
\newcommand{\todo}[1]{\bigskip \noindent \emph{todo: #1}}
%\newcommand{\todo}[1]{}
\newcommand{\semantics}[1]{\llbracket #1 \rrbracket}

% church-style (explicit) abstraction. We adjust the spacing around the colon and dot
\newcommand{\church}[4]{#1 #2\!:\!#3\,.\,#4}

\newcommand{\curry}[3]{#1 #2\,.\,#3}

% something is of type something  x : A, we adjust the spacing
\newcommand{\oftype}[2]{#1\!:\!#2}

% === END custom commands

\begin{document}

\title{Proposal for SPL Part 4}
\author{Markus Klinik (s4220315)}
\maketitle

\begin{abstract}

This document proposes an extension to the SPL compiler developed for the course
NWI-IMC004 Compiler Construction.

\end{abstract}

\section{Records and structural subtyping}

The goal of this endeavor is to implement a flexible form of subtyping for
records, called \emph{structural subtyping}.  A \emph{record} is a data
structure which groups several values into a single value.  The individual
components, called \emph{fields}, are accessed via names, called \emph{labels}.
A record type $\beta$ is a subtype of record type $\alpha$ iff $\beta$ has at
least all the fields which $\alpha$ has, and possibly more.  The inverse
relation is called \emph{supertype}. No explicit type declarations indicating
subtype relationships are needed.

Record types are subject to type inference and polymorphism.

Functions taking records of type $\alpha$ as arguments can be given any value of
type $\beta$, as long as $\beta$ is a subtype of $\alpha$.  The same goes for
return values.  For example, a function which uses only the fields, say,
\emph{x} and \emph{y} of type \emph{Int} of a record, can be given any record
which has at least these two fields also of type \emph{Int}.

The order of fields in a record should not matter, nor can it matter, for there
are easily conceived examples of different subtypes where the
compiler cannot order the fields in a consistent manner.  Consistent meaning
that the same lookup code can be used for supertypes, as it is the case for
systems with nominal subtyping, like Java.  Getting the code for field lookup
right will be the main challenge for the code generation phase.

\section{Subgoals}

The first step would be to implement records without subtyping, that is the
types of records always have to match exactly.  Records can be created without
declaring their type beforehand.  The type of a record forming expression is
completely determined by its structure.

For this, the grammar has to be adapted in the following ways. We need to

\begin{itemize}

  \item extend the syntax of types to allow specifying records,

  \item extend the syntax of expressions to allow creating records, and

  \item extend the syntax of expressions to allow accessing fields of records.
  
\end{itemize}

Once the
first step is established, the type system can be extended to feature subtyping
as a second step.  No modifications in the back end should be needed.  The type
system should just admit more programs where records of different types are used
in places where it is safe.

\section{Examples}

The following program fragment is a function which creates a new record with
fields \emph{x} and \emph{y} with the values of the given record, offset by the
given amount \emph{diff}.

\begin{verbatim}
{Int x, Int y} translate({Int x, Int y} val, Int diff)
{
  return { Int x = val.x + diff, Int y = val.y + diff }
}
\end{verbatim}

Next is a polymorphic function which extracts the field named \emph{foo} from
every record that has one, no matter what the type of \emph{foo} is.

\begin{verbatim}
a getFoo(b val) {
  return val.foo;
}
\end{verbatim}

The following example shows a function that extracts all fields \emph{x} from a
list of records and collects them into a new list in reverse order.

\begin{verbatim}
[a] getXs([{Int x}] recs) {
  [a] acc = [];
  while( !isEmpty(recs) ) {
    acc = head(recs).x : acc;
    recs = tail(recs);
  }
  return acc;
}
\end{verbatim}

\section{Limitations}

The main points of interest here are subtyping and type inference of function
arguments.  To not overstress the time frame of the course, we intend to cut
down on any point not related to this.  Issues like the following come to mind.

\begin{itemize}

  \item Fields of records can be read, but not assigned to.

  \item Labels are not first-class.  Accessing a field is a static syntactic
  construct.  For example, the following will not be allowed.

  \begin{verbatim}
  c project(a x, b y) {
    return x.y;
  }
  \end{verbatim}

\end{itemize}

\bibliographystyle{plain}
\bibliography{computer_science}

\end{document}

% vim: textwidth=80
