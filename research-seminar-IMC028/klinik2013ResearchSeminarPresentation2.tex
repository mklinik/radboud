\documentclass{beamer}

\beamertemplatenavigationsymbolsempty

\mode<presentation>
{
  \usetheme{default}
}

\usepackage[english]{babel}
\usepackage[latin1]{inputenc}
\usepackage{bussproofs}

% needs debian package texlive-math-extra
\usepackage{stmaryrd} % for \llbracket, \rrbracket

\usepackage{times}
\usepackage[T1]{fontenc}
% Or whatever. Note that the encoding and the font should match. If T1
% does not look nice, try deleting the line with the fontenc.

\usepackage{tikz}
\usetikzlibrary{positioning}
\usetikzlibrary{calc}

\title
{A Domain for the\\
Coinductive~Natural~Numbers\\
Part 2}

\author
{Markus~Klinik}

\institute[Radboud University Nijmegen] % (optional, but mostly needed)
{
  Radboud University Nijmegen
}

\date
{MFoCS Research Seminar 2012}


\newcommand{\arr}{\rightarrow}
\newcommand{\Arr}{\Rightarrow}

\begin{document}

\begin{frame}
  \titlepage
\end{frame}

\begin{frame}{Outline}
  \tableofcontents
\end{frame}

\begin{frame}{Fixed Points of Functors}

\begin{itemize}%[<+->]
  \item $\mathbf{Set}$-Functors take sets to sets
  \item Identity, powerset, cartesian product, function space
  \item Fixed point: $f(x) = x$
  \item Two sets are the same if \ldots
  \item We are interested in isomorphic sets
    \begin{itemize}
      \item $FX \sim X$
    \end{itemize}
\end{itemize}

\end{frame}


\begin{frame}{Examples}

\begin{itemize}
  \item Cantor: $X \nsim PX$
  \item What about $1 \dot\cup X$?
\end{itemize}

\end{frame}


\begin{frame}{1?}

Let's try $X = 1$, i.e. $\{*\}$

\begin{center}

$X = \{*\}$

\bigskip

$1 \dot\cup X = \{\langle 0, * \rangle, \langle 1, * \rangle\}$

\end{center}

\end{frame}


\begin{frame}{$\mathbb{N}$?}

Let's try $X = \mathbb{N}$

\begin{center}

$\{1, 2, 3, \ldots \}$

\bigskip

$\{\langle 0, * \rangle
, \langle 1, 1 \rangle
, \langle 1, 2 \rangle
, \langle 1, 3 \rangle
, \ldots
\}$

\end{center}

\end{frame}


\newcommand{\picCpo}[1]{
\node (bottom#1) {$\bot$};
\node (top) [above=of bottom#1] {};
\node (topleft) [left=5mm of top] {};
\node (topright) [right=5mm of top] {};
\draw (bottom#1) -- (topleft.center) -- (topright.center) -- (bottom#1);
}


\begin{frame}{The Coinductive Natural Numbers}

\begin{itemize}

  \item We are not in $\mathbf{Set}$, we are in $\mathbf{Cpo}$
  \item Our sets have structure that functors must preserve
  \item Here: $(X, \sqsubseteq)$ where
    \begin{itemize}
      \item $\sqsubseteq$ is a partial ordering
      \item $\sqsubseteq$ is chain-complete
      \item $(X, \sqsubseteq)$ has a least element
    \end{itemize}

\end{itemize}

%\onslide<+->

\begin{center}
\begin{tikzpicture}
\picCpo{}
\end{tikzpicture}
\end{center}

\end{frame}


\begin{frame}{The Separated Sum Functor}

\begin{center}
\begin{tikzpicture}

  \onslide<+->

  \picCpo{l}

  \begin{scope}[xshift=2.5cm]
  \picCpo{r}
  \end{scope}

  \onslide<+->

  \path (1.25, -1) node (newBottom) {$\bot$};
  \draw (bottoml) -- (newBottom) -- (bottomr);

  \onslide<+->

  \begin{scope}[xshift=5cm]
  \picCpo{rr}
  \end{scope}

  \onslide<+->

  \path (2.5, -2.3) node (newNewBottom) {$\bot$};
  \draw (newBottom) -- (newNewBottom) -- (bottomrr);

\end{tikzpicture}
\end{center}

\end{frame}


\begin{frame}{The I-Place Separated Sum Functor}
\end{frame}


\end{document}
