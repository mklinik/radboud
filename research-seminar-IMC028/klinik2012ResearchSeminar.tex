\documentclass[a4paper]{article}
\usepackage{mathptmx}
%\usepackage[square]{natbib}
\begin{document}

\title{A domain for the New Naturals}
\author{Markus Klinik}
\maketitle

\begin{abstract}
Lorem ipsum.
\end{abstract}

\section{What are the New Naturals?}

\begin{itemize}
\item Explain inductive data types, F-algebras.
\item Give Examples.
\item Explain coinductive data types, F-coalgebras.
\item Give Examples.
\item The New Naturals are the coinductive data type associated with the
functor $F(X) = 1 + X + X$.
\end{itemize}

\section{What is a domain?}

\begin{itemize}
\item A domain is a CPO with a least element.
\item And what is a CPO?  A partial order where every chain has a least upper bound.

\item Finding a domain for the New Naturals means finding a suitable CPO
together with a mapping from the New Naturals to that CPO.

\item We then need to prove that the given construction is indeed a CPO with
least element.

\end{itemize}

\section{Why do we want a domain for the New Naturals?}

\begin{itemize}

\item We want to use the New Naturals in a programming language.

\item We want to be able to reason about this programming language.

\item One way to reason about a programming language is to give it a
denotational semantics and then reason about that.

\item In order to do this, every language construct needs a corresponding
construct in the denotational semantics.

\item Just because we define something, doesn't mean it exists.  Think halting
problem, think Russel's paradox. If we manage to let programs denote domains,
we can apply Tarski's fixed point theorem to be sure that every program has
indeed a valid denotation.

\end{itemize}

\section{A domain for the New Naturals}

Here be dragons.

\section{Further directions}

\begin{itemize}
\item Embed the New Naturals into some coinductive PCF.
\item Analyze if the domain composes nicely (or at all) with the other
denotations.
\item Can we generalize the construction of the domain to arbitrary
F-coalgebras?
\end{itemize}

\section{Bibliographic notes}

Lorem ipsum \cite{Pierce1991} \cite{Gunter1992} \cite{Bird1997}
\cite{Mitchell1996} \cite{Allison1986} \cite{Capretta2002}

\bibliographystyle{plain}
\bibliography{computer_science}
\end{document}
